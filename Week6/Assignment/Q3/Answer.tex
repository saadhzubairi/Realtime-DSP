\documentclass[11pt]{article}
\usepackage{graphicx}
\usepackage{amsmath}
\usepackage{hyperref}
\usepackage{listings}


\topmargin=-0.45in
\evensidemargin=0in
\oddsidemargin=0in
\textwidth=6.5in
\textheight=9.0in
\headsep=0.25in
\renewcommand{\thesubsection}{\alph{subsection})}
\renewcommand\lstlistingname{Snippet}
\renewcommand\lstlistlistingname{}

\usepackage{xcolor}
\definecolor{codegreen}{rgb}{0,0.6,0}
\definecolor{codegray}{rgb}{0.5,0.5,0.5}
\definecolor{codeorange}{rgb}{1,0.49,0}
\definecolor{backcolour}{rgb}{0.95,0.95,0.96}

\lstdefinestyle{mystyle}{
    backgroundcolor=\color{backcolour},   
    commentstyle=\color{codegray},
    keywordstyle=\color{codeorange},
    numberstyle=\tiny\color{codegray},
    stringstyle=\color{codegreen},
    basicstyle=\ttfamily\footnotesize,
    breakatwhitespace=false,         
    breaklines=true,                 
    captionpos=b,                    
    keepspaces=true,                 
    numbers=left,                    
    numbersep=5pt,                  
    showspaces=false,                
    showstringspaces=false,
    showtabs=false,                  
    tabsize=2,
    xleftmargin=10pt,
}

\lstset{style=mystyle}
% Title and Author Customization

% --------------------
% Start from here
% --------------------

\title{
    \vspace{3em}
    \textbf{Digital Signal Processing Lab}\\
    Demo 54 - Exercise 2 (AM effect with live spectrum)
    \vspace{1em}
}
\author{
    Saad Zubairi \\ 
    shz2020 \\
    \vspace{1em}
}
\vspace{1em}
\date{\today}

\begin{document}
\maketitle	

\pagebreak

% --------------------
% Body
% --------------------


\section*{Solution}

To implement the AM effect, we reused the same real-time microphone acquisition and plotting framework from Question 1, replacing the high-pass filter with a cosine-based amplitude modulator.\\
\\
This setup multiplies the incoming audio samples by a high-frequency carrier signal, shifting the spectrum to higher frequencies while maintaining real-time playback and visualization.

\subsection*{Overview of Modifications}
\begin{itemize}
    \item \textbf{Carrier Initialization}: A 1 kHz cosine carrier was chosen to clearly illustrate frequency translation in the spectrum. The carrier parameters and phase increment were initialized as follows
    
    \begin{lstlisting}[language=python, label={lst:code}, breaklines=true]
    f_carrier = 1000.0      # Hz
    phase = 0.0
    phase_inc = 2.0 * np.pi * f_carrier / RATE
    \end{lstlisting}    

    \item \textbf{Amplitude Modulation Process}: For each input block from the microphone, the signal was multiplied by the carrier, producing an amplitude-modulated output signal.
\\The per-block processing loop is identical to the high-pass example except for the multiplication step
    
    \begin{lstlisting}[language=python, label={lst:code}, breaklines=true]
    carrier = np.cos(phase + phase_inc * np.arange(BLOCKLEN))
    y_block = x_block * carrier
    phase = (phase + phase_inc * BLOCKLEN) % (2.0 * np.pi)
    y_play = np.clip(y_block, -32768, 32767).astype('int16')
    stream.write(y_play.tobytes(), BLOCKLEN)
    \end{lstlisting}    
    
    \item \textbf{Matplotlib visualization}:The Matplotlib animation setup remains identical to Question 1:

    \begin{itemize}
        \item Input Signal (time domain)
        \item Spectrum of Input (with frequency response 100)
        \item Output Signal
        \item Spectrum of Output
    \end{itemize}
\end{itemize}
\pagebreak
\section*{Addendum: Full implementation}

\begin{lstlisting}[language=python, label={lst:code}, breaklines=true, caption={example code}]
import pyaudio
import matplotlib
from matplotlib import pyplot
from matplotlib import animation
import numpy as np

matplotlib.use('TkAgg')
print('The matplotlib backend is %s' % pyplot.get_backend())
WIDTH = 2            # bytes per sample
CHANNELS = 1         # mono
RATE = 8000          # frames per second
BLOCKLEN = 512       # block length in samples
# BLOCKLEN = 256
print('Block length: %d' % BLOCKLEN)
print('Duration of block in milliseconds: %.1f' % (1000.0 * BLOCKLEN / RATE))

p = pyaudio.PyAudio()
print("Default input device:", p.get_default_input_device_info()["name"])
PA_FORMAT = p.get_format_from_width(WIDTH)
stream = p.open(
    format=PA_FORMAT,
    channels=CHANNELS,
    rate=RATE,
    input=True,          
    output=True,         
    input_device_index=None, 
    output_device_index=None,
    frames_per_buffer=BLOCKLEN
)

# high pass filter diff equation
""" fc_hz = 0.1 * RATE   
RC = 1.0 / (2.0 * np.pi * fc_hz)
T = 1.0 / RATE
alpha = RC / (RC + T)

x_prev = 0.0
y_prev = 0.0 """



# figure prep
fig1 = pyplot.figure(1)
fig1.set_size_inches((12, 7)) 

ax_x = fig1.add_subplot(2, 2, 1)
ax_X = fig1.add_subplot(2, 2, 2)
ax_y = fig1.add_subplot(2, 2, 3)
ax_Y = fig1.add_subplot(2, 2, 4)

t = np.arange(BLOCKLEN) * (1000.0 / RATE)  
x = np.zeros(BLOCKLEN)                     
X = np.fft.rfft(x)                         
f_X = np.arange(X.size) * RATE / BLOCKLEN  

# Precompute HPF frequency response curve for plotting 
# H(e^jw) = alpha * (1 - e^{-jw}) / (1 - alpha * e^{-jw})
""" w = 2.0 * np.pi * (np.linspace(0, RATE/2, num=X.size) / RATE)  # rad/sample
ejw = np.exp(-1j * w)
H = alpha * (1.0 - ejw) / (1.0 - alpha * ejw)
f_H = np.linspace(0, RATE/2, num=X.size)
 """
# input signal plot
[g_x] = ax_x.plot([], [])
ax_x.set_ylim(-10000, 10000)
ax_x.set_xlim(0, 1000.0 * BLOCKLEN / RATE)
ax_x.set_xlabel('Time (milliseconds)')
ax_x.set_title('Input signal')

# input spectrum plot (+ HPF response x100)
[g_X] = ax_X.plot([], [])
#[g_H] = ax_X.plot(f_H, 100.0 * np.abs(H), label='Frequency response (x100)', color='green')
ax_X.set_xlim(0, RATE/2)
ax_X.set_ylim(0, 300)  # matches the visual scale in your screenshot
ax_X.set_title('Spectrum of input signal')
ax_X.set_xlabel('Frequency (Hz)')
ax_X.legend()

# AM params
f_carrier = 1000.0  # Hz (carrier frequency)
t_block = np.arange(BLOCKLEN) / RATE
phase = 0.0
phase_inc = 2.0 * np.pi * f_carrier / RATE

# output signal plot
[g_y] = ax_y.plot([], [])
ax_y.set_ylim(-10000, 10000)
ax_y.set_xlim(0, 1000.0 * BLOCKLEN / RATE)
ax_y.set_xlabel('Time (milliseconds)')
ax_y.set_title('Output signal')

# output spectrum plot
[g_Y] = ax_Y.plot([], [])
ax_Y.set_xlim(0, RATE/2)
ax_Y.set_ylim(0, 500)   # matches the visual scale in your screenshot
ax_Y.set_title('Spectrum of output signal')
ax_Y.set_xlabel('Frequency (Hz)')

fig1.tight_layout()

def my_init():
    g_x.set_xdata(t)
    g_x.set_ydata(x)
    g_y.set_xdata(t)
    g_y.set_ydata(x)
    g_X.set_xdata(f_X)
    g_X.set_ydata(np.abs(X))
    g_Y.set_xdata(f_X)
    g_Y.set_ydata(np.abs(X))
    return (g_x, g_y, g_X, g_Y)

def my_update(i):
    global phase

    # read audio input
    signal_bytes = stream.read(BLOCKLEN, exception_on_overflow=False)
    x_block = np.frombuffer(signal_bytes, dtype='int16').astype(np.float64)

    # --- apply amplitude modulation (AM) ---
    carrier = np.cos(phase + phase_inc * np.arange(BLOCKLEN))
    y_block = x_block * carrier
    phase = (phase + phase_inc * BLOCKLEN) % (2.0 * np.pi)

    # --- spectra ---
    Xk = np.fft.rfft(x_block) / BLOCKLEN
    Yk = np.fft.rfft(y_block) / BLOCKLEN

    # --- update plots ---
    g_x.set_ydata(x_block)
    g_y.set_ydata(y_block)
    g_X.set_ydata(np.abs(Xk))
    g_Y.set_ydata(np.abs(Yk))

    # --- playback ---
    y_play = np.clip(y_block, -32768, 32767).astype('int16')
    stream.write(y_play.tobytes(), BLOCKLEN)

    return (g_x, g_y, g_X, g_Y)

my_anima = animation.FuncAnimation(
    fig1,
    my_update,
    init_func=my_init,
    interval=10,          # milliseconds
    blit=True,
    cache_frame_data=False,
    repeat=False
)
pyplot.show()

stream.stop_stream()
stream.close()
p.terminate()
print('* Finished')
\end{lstlisting}    

    
\end{document}