\documentclass[11pt]{article}
\usepackage{graphicx}
\usepackage{amsmath}
\usepackage{hyperref}
\usepackage{listings}


\topmargin=-0.45in
\evensidemargin=0in
\oddsidemargin=0in
\textwidth=6.5in
\textheight=9.0in
\headsep=0.25in
\renewcommand{\thesubsection}{\alph{subsection})}
\renewcommand\lstlistingname{Snippet}
\renewcommand\lstlistlistingname{}

\usepackage{xcolor}
\definecolor{codegreen}{rgb}{0,0.6,0}
\definecolor{codegray}{rgb}{0.5,0.5,0.5}
\definecolor{codeorange}{rgb}{1,0.49,0}
\definecolor{backcolour}{rgb}{0.95,0.95,0.96}

\lstdefinestyle{mystyle}{
    backgroundcolor=\color{backcolour},   
    commentstyle=\color{codegray},
    keywordstyle=\color{codeorange},
    numberstyle=\tiny\color{codegray},
    stringstyle=\color{codegreen},
    basicstyle=\ttfamily\footnotesize,
    breakatwhitespace=false,         
    breaklines=true,                 
    captionpos=b,                    
    keepspaces=true,                 
    numbers=left,                    
    numbersep=5pt,                  
    showspaces=false,                
    showstringspaces=false,
    showtabs=false,                  
    tabsize=2,
    xleftmargin=10pt,
}

\lstset{style=mystyle}
% Title and Author Customization

% --------------------
% Start from here
% --------------------

\title{
    \vspace{3em}
    \textbf{Digital Signal Processing Lab}\\
    Demo 10 - Exercise 6 (Vibrato with non-sinusoidal LFO)
    \vspace{1em}
}
\author{
    Saad Zubairi \\ 
    shz2020 \\
    \vspace{1em}
}
\vspace{1em}
\date{September 17th, 2025}

\begin{document}
\maketitle	

\pagebreak

% --------------------
% Body
% --------------------


\section*{Solution}

Solution Text

% --------------------
% Code Blocks example
% --------------------

\section*{Solution}
In the solution, the sinusoidal LFO was replaced with a \textbf{triangle wave LFO}. 
Apart from this, the program will now also \textbf{save the processed output signal as a WAV file}, while still playing the signal through the audio device. 

\subsection*{Code Changes and Additions Made}
Below are the changes and additions made to the original file.

\begin{itemize}
    \item Added code to open a new wave file for writing:
    \begin{lstlisting}[language=python, caption={Open output wave file}]
# Also save to file
output_wav = wave.open("output_vibrato.wav", 'w')
output_wav.setnchannels(1)
output_wav.setsampwidth(2)
output_wav.setframerate(RATE)
    \end{lstlisting}
    
    \item Added a \textbf{triangle wave LFO} instead of sinusoid:
    \begin{lstlisting}[language=python, caption={Triangle wave LFO}]
# -------- LFO: Triangle wave instead of sinusoid --------
phase = (n * f0 / RATE) % 1.0
if phase < 0.5:
    lfo = (phase * 4.0 - 1.0)   # -1 to +1 rising
else:
    lfo = (3.0 - phase * 4.0)   # +1 to -1 falling
# --------------------------------------------------------
kr = i1 + Wd * lfo
    \end{lstlisting}
    
    \item Each computed output frame is now also written to the file:
    \begin{lstlisting}[language=python, caption={Write frames to file}]
output_wav.writeframes(output_bytes)
    \end{lstlisting}
    
    \item Finally, closed the output file at the end:
    \begin{lstlisting}[language=python, caption={Close output file}]
output_wav.close()
    \end{lstlisting}
\end{itemize}
\pagebreak
\subsection*{Addendum: Full Code}
    
\begin{lstlisting}[language=python, caption={Full code}]
# play_vibrato_interpolation.py
# Reads a specified wave file (mono) and plays it with a vibrato effect.
# (Time-varying delay using interpolation)
# Modified: LFO is now a triangle wave, and output is saved as WAV

import pyaudio
import wave
import struct
import math
from myfunctions import clip16

# wavfile = 'decay_cosine_mono.wav'
wavfile = 'author.wav'
# wavfile = 'cosine_300_hz.wav'

print('Play the wave file: %s.' % wavfile)

# Open wave file
wf = wave.open(wavfile, 'rb')

# Read wave file properties
RATE        = wf.getframerate()
WIDTH       = wf.getsampwidth()
LEN         = wf.getnframes()
CHANNELS    = wf.getnchannels()

print('The file has %d channel(s).'         % CHANNELS)
print('The file has %d frames/second.'      % RATE)
print('The file has %d frames.'             % LEN)
print('The file has %d bytes per sample.'   % WIDTH)

# Vibrato parameters
f0 = 2          # LFO frequency in Hz
W = 0.015       # Sweep width (seconds)
Wd = W * RATE   # in samples

# Buffer
BUFFER_LEN = 1024
buffer = BUFFER_LEN * [0]

kr = 0
i1 = kr
kw = int(0.5 * BUFFER_LEN)

print('The buffer is %d samples long.' % BUFFER_LEN)

# Output stream
p = pyaudio.PyAudio()
stream = p.open(format      = pyaudio.paInt16,
                channels    = 1,
                rate        = RATE,
                input       = False,
                output      = True )

# save to file
output_wav = wave.open("output_vibrato.wav", 'w')
output_wav.setnchannels(1)
output_wav.setsampwidth(2)
output_wav.setframerate(RATE)

print ('* Playing...')

for n in range(0, LEN):

    input_bytes = wf.readframes(1)
    x0, = struct.unpack('h', input_bytes)

    kr_prev = int(math.floor(kr))
    frac = kr - kr_prev
    kr_next = kr_prev + 1
    if kr_next == BUFFER_LEN:
        kr_next = 0

    y0 = (1-frac) * buffer[kr_prev] + frac * buffer[kr_next]

    buffer[kw] = x0

    # -------- LFO: Triangle wave instead of sinusoid --------
    # Normalized phase: goes from 0 to 1 each cycle
    phase = (n * f0 / RATE) % 1.0
    if phase < 0.5:
        lfo = (phase * 4.0 - 1.0)   # -1 to +1 rising
    else:
        lfo = (3.0 - phase * 4.0)   # +1 to -1 falling
    # --------------------------------------------------------

    kr = i1 + Wd * lfo
    if kr >= BUFFER_LEN:
        kr = kr - BUFFER_LEN
    if kr < 0:
        kr = kr + BUFFER_LEN

    i1 = i1 + 1
    if i1 >= BUFFER_LEN:
        i1 = i1 - BUFFER_LEN

    kw = kw + 1
    if kw == BUFFER_LEN:
        kw = 0

    output_bytes = struct.pack('h', int(clip16(y0)))
    stream.write(output_bytes)
    output_wav.writeframes(output_bytes)

print('* Finished')

stream.stop_stream()
stream.close()
p.terminate()
wf.close()
output_wav.close()
\end{lstlisting}

\end{document}