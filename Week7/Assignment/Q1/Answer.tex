\documentclass[11pt]{article}
\usepackage{graphicx}
\usepackage{amsmath}
\usepackage{hyperref}
\usepackage{listings}


\topmargin=-0.45in
\evensidemargin=0in
\oddsidemargin=0in
\textwidth=6.5in
\textheight=9.0in
\headsep=0.25in
\renewcommand{\thesubsection}{\alph{subsection})}
\renewcommand\lstlistingname{Snippet}
\renewcommand\lstlistlistingname{}

\usepackage{xcolor}
\definecolor{codegreen}{rgb}{0,0.6,0}
\definecolor{codegray}{rgb}{0.5,0.5,0.5}
\definecolor{codeorange}{rgb}{1,0.49,0}
\definecolor{backcolour}{rgb}{0.95,0.95,0.96}

\lstdefinestyle{mystyle}{
    backgroundcolor=\color{backcolour},   
    commentstyle=\color{codegray},
    keywordstyle=\color{codeorange},
    numberstyle=\tiny\color{codegray},
    stringstyle=\color{codegreen},
    basicstyle=\ttfamily\footnotesize,
    breakatwhitespace=false,         
    breaklines=true,                 
    captionpos=b,                    
    keepspaces=true,                 
    numbers=left,                    
    numbersep=5pt,                  
    showspaces=false,                
    showstringspaces=false,
    showtabs=false,                  
    tabsize=2,
    xleftmargin=10pt,
}

\lstset{style=mystyle}
% Title and Author Customization

% --------------------
% Start from here
% --------------------

\title{
    \vspace{3em}
    \textbf{Digital Signal Processing Lab}\\
    Demo 20 - Guitar
    \vspace{1em}
}
\author{
    Saad Zubairi \\ 
    shz2020 \\
    \vspace{1em}
}
\vspace{1em}
\date{\today}

\begin{document}
\maketitle	

\pagebreak

% --------------------
% Body
% --------------------


\section*{Solution}

Our program implements the Karplus-Strong plucked-string algorithm in real time using Python and PyAudio.\\
A circular buffer is initialized with white noise to represent the pluck excitation. Each output sample is generated by reading one value from the buffer, then replacing it with the average of the first two samples multiplied by a damping factor 
K to simulate energy loss. \\
The code streams each sample directly to the audio device, ensuring real-time playback without using block processing or \texttt{lfilter}.\\

Key points:
\begin{itemize}
    \item Sampling rate: Fs=8000 Hz
    \item Damping factor $K\approx0.998$ controls sustain
    \item Frequency determined by buffer size textbf{N} 
    \item Circular buffer implementation ensures efficient real-time behavior
\end{itemize}

This setup synthesizes a natural-sounding plucked string tone entirely in Python using direct sample-by-sample processing.
\pagebreak
\section*{Addendum}Full Code
\begin{lstlisting}[language=python, label={lst:code}, breaklines=true, caption={Complete implementation}]
import numpy as np
import pyaudio
import struct

def karplus_strong(
    frequency=150.0, 
    Fs=8000, 
    K=0.998):
    
    duration=10*50/frequency
    #print(duration)
    print("* Playing")
    N = int(Fs / frequency)
    buffer = np.random.uniform(-1, 1, N)
    p = pyaudio.PyAudio()
    stream = p.open(format=pyaudio.paInt16, channels=1, rate=Fs, output=True)
    
    num_samples = int(duration * Fs)
    MAXVALUE = 2**15 - 1
    
    # circular buffer index
    idx = 0

    for i in range(num_samples):
        y = buffer[idx]
        next_idx = (idx + 1) % N
        avg = 0.5 * (buffer[idx] + buffer[next_idx])
        buffer[idx] = K * avg  # feedback
        idx = next_idx
        # convert to bytes and play
        sample = struct.pack('h', int(y * MAXVALUE))
        stream.write(sample)

    stream.stop_stream()
    stream.close()
    p.terminate()
    print("* Finished")

karplus_strong(110,K=0.998)
\end{lstlisting}    
    
\end{document}