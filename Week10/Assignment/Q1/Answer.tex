\documentclass[11pt]{article}
\usepackage{graphicx}
\usepackage{amsmath}
\usepackage{hyperref}
\usepackage{listings}


\topmargin=-0.45in
\evensidemargin=0in
\oddsidemargin=0in
\textwidth=6.5in
\textheight=9.0in
\headsep=0.25in
\renewcommand{\thesubsection}{\alph{subsection})}
\renewcommand\lstlistingname{Snippet}
\renewcommand\lstlistlistingname{}

\usepackage{xcolor}
\definecolor{codegreen}{rgb}{0,0.6,0}
\definecolor{codegray}{rgb}{0.5,0.5,0.5}
\definecolor{codeorange}{rgb}{1,0.49,0}
\definecolor{backcolour}{rgb}{0.95,0.95,0.96}

\lstdefinestyle{mystyle}{
    backgroundcolor=\color{backcolour},   
    commentstyle=\color{codegray},
    keywordstyle=\color{codeorange},
    numberstyle=\tiny\color{codegray},
    stringstyle=\color{codegreen},
    basicstyle=\ttfamily\footnotesize,
    breakatwhitespace=false,         
    breaklines=true,                 
    captionpos=b,                    
    keepspaces=true,                 
    numbers=left,                    
    numbersep=5pt,                  
    showspaces=false,                
    showstringspaces=false,
    showtabs=false,                  
    tabsize=2,
    xleftmargin=10pt,
}

\lstset{style=mystyle}
% Title and Author Customization

% --------------------
% Start from here
% --------------------

\title{
    \vspace{3em}
    \textbf{Digital Signal Processing Lab}\\
    Demo 22 - Exercise 1 (Video processing)
    \vspace{1em}
}
\author{
    Saad Zubairi \\ 
    shz2020 \\
    \vspace{1em}
}
\vspace{1em}
\date{\today}

\begin{document}
\maketitle	

\pagebreak

% --------------------
% Body
% --------------------

\section*{Solution}

I started from the demo file \texttt{D5 - video operations/blur\_video.py}. That script already opened the webcam, blurred each frame, handled the keyboard shortcuts, and saved a snapshot when \texttt{p} was pressed. To convert it into an edge detector I only had to swap the filtering step for a high-pass filter:
\begin{itemize}
    \item Replaced the Gaussian blur with a $3\times3$ Laplacian-style kernel so high-frequency components (edges) are emphasized instead of suppressed.
    \item Applied \texttt{cv2.filter2D} with that kernel.
\end{itemize}
The rest of the loop (capture, display, save) stayed untouched.

\begin{lstlisting}[language=python, label={lst:q1diff}, breaklines=true, caption={Key change for edge detection}]
kernel = np.array([[-1, -1, -1],
                   [-1,  8, -1],
                   [-1, -1, -1]])

[ok, frame] = cap.read()
frame = cv2.filter2D(frame, -1, kernel)
cv2.imshow('Live video (edges)', frame)
\end{lstlisting}

\section*{Screenshots}
\begin{center}
    \includegraphics[scale=0.5]{Example.png}
\end{center}

\pagebreak
\section*{Addendum: Full implementation}

\begin{lstlisting}[language=python, label={lst:q1full}, breaklines=true, caption={Full implementation}]
# blur_video.py
# Demonstrates 2D spatial filtering

import numpy as np 
import cv2

cap = cv2.VideoCapture(0)

print("Switch to video window. Then press 'p' to save image, 'q' to quit")

# High-pass kernel highlights edges instead of smoothing
kernel = np.array([[-1, -1, -1],
                   [-1,  8, -1],
                   [-1, -1, -1]])

while True:

    [ok, frame] = cap.read()          # Read one frame

    # Use 2D filtering with the high-pass kernel to emphasize edges
    frame = cv2.filter2D(frame, -1, kernel)

    cv2.imshow('Live video (edges)', frame)

    key = cv2.waitKey(1)
    # key = key & 0xFF      # (May not be necessary)

    if key == ord('p'):
        cv2.imwrite('edges.jpg', frame)              
        
    if key == ord('q'):
        break

cap.release()
cv2.destroyAllWindows()
\end{lstlisting}

    
\end{document}
