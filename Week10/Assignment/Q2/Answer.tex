\documentclass[11pt]{article}
\usepackage{graphicx}
\usepackage{amsmath}
\usepackage{hyperref}
\usepackage{listings}


\topmargin=-0.45in
\evensidemargin=0in
\oddsidemargin=0in
\textwidth=6.5in
\textheight=9.0in
\headsep=0.25in
\renewcommand{\thesubsection}{\alph{subsection})}
\renewcommand\lstlistingname{Snippet}
\renewcommand\lstlistlistingname{}

\usepackage{xcolor}
\definecolor{codegreen}{rgb}{0,0.6,0}
\definecolor{codegray}{rgb}{0.5,0.5,0.5}
\definecolor{codeorange}{rgb}{1,0.49,0}
\definecolor{backcolour}{rgb}{0.95,0.95,0.96}

\lstdefinestyle{mystyle}{
    backgroundcolor=\color{backcolour},   
    commentstyle=\color{codegray},
    keywordstyle=\color{codeorange},
    numberstyle=\tiny\color{codegray},
    stringstyle=\color{codegreen},
    basicstyle=\ttfamily\footnotesize,
    breakatwhitespace=false,         
    breaklines=true,                 
    captionpos=b,                    
    keepspaces=true,                 
    numbers=left,                    
    numbersep=5pt,                  
    showspaces=false,                
    showstringspaces=false,
    showtabs=false,                  
    tabsize=2,
    xleftmargin=10pt,
}

\lstset{style=mystyle}
% Title and Author Customization

% --------------------
% Start from here
% --------------------

\title{
    \vspace{3em}
    \textbf{Digital Signal Processing Lab}\\
    Demo 22 - Exercise 3 (Video processing)
    \vspace{1em}
}
\author{
    Saad Zubairi \\ 
    shz2020 \\
    \vspace{1em}
}
\vspace{1em}
\date{\today}

\begin{document}
\maketitle	

\pagebreak

% --------------------
% Body
% --------------------

\section*{Solution}

For this, I used the color demo \texttt{D7 - color operations/find\_blue\_in\_image.py} as the base. The entire flow (load the sample fruit image, convert to HSV, build a mask, display/save the results) was untouched of course, with the following changes added:
\begin{itemize}
    \item Replaced the BGR reference color with pure green and reused the same hue-based bandpass idea.
    \item Clamped the hue limits inside the valid HSV range and renamed the saved images to \\
    \texttt{fruit\_mask\_green.jpg} / \texttt{fruit\_green.jpg}.
\end{itemize}
The HSV mask isolates green pixels, and \texttt{cv2.bitwise\_and} paints only those pixels in the detected image.

\begin{lstlisting}[language=python, label={lst:q2diff}, breaklines=true, caption={Key changes for green detection}]
img = cv2.imread('fruit.jpg', 1)
green = np.uint8([[[0, 255, 0]]])
green_hsv = cv2.cvtColor(green, cv2.COLOR_BGR2HSV)
h = green_hsv[0,0,0]
lower = np.array([max(h-20, 0), 50, 50])
upper = np.array([min(h+20, 179), 255, 255])
mask = cv2.inRange(img_hsv, lower, upper)
output = cv2.bitwise_and(img, img, mask = mask)
\end{lstlisting}

\section*{Screenshots}

\begin{center}
    \includegraphics[scale=0.25]{Result.png}
\end{center}

\pagebreak
\section*{Addendum: Full implementation}

\begin{lstlisting}[language=python, label={lst:q2full}, breaklines=true, caption={Full implementation}]
import cv2
import numpy as np 

img = cv2.imread('fruit.jpg', 1)   
# Convert to different color space
img_hsv = cv2.cvtColor(img, cv2.COLOR_BGR2HSV)

print(type(img_hsv))
print(img_hsv.shape)
print(img_hsv.dtype)

blue = np.uint8([[[0, 255, 0]]])   # 3D array describing green in BGR
blue_hsv = cv2.cvtColor(blue, cv2.COLOR_BGR2HSV)
h = blue_hsv[0,0,0]
print('Blue in HSV color space:', blue_hsv)
print('Hue = ', h)   # see that h = 120

lower = np.array([max(h-20, 0), 50, 50])
upper = np.array([min(h+20, 179), 255, 255])
print('lower = ', lower)
print('upper = ', upper)

# Determine binary mask
blue_mask = cv2.inRange(img_hsv, lower, upper)

# Apply mask to color image
output = cv2.bitwise_and(img, img, mask = blue_mask)

# Show images:
cv2.imshow('Original image', img)
cv2.imshow('Mask', blue_mask)
cv2.imshow('Segmented image', output)

print('Switch to images. Then press any key to stop')

cv2.waitKey(0)
cv2.destroyAllWindows()

# Write the image to a file
cv2.imwrite('fruit_mask_green.jpg', blue_mask)   
cv2.imwrite('fruit_green.jpg', output)   


# Reference
# http://docs.opencv.org/3.2.0/df/d9d/tutorial_py_colorspaces.html
\end{lstlisting}

    
\end{document}
