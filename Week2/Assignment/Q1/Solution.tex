\documentclass[11pt]{article}
\usepackage{graphicx}
\usepackage{amsmath}
\usepackage{hyperref}
\usepackage{listings}


\topmargin=-0.45in
\evensidemargin=0in
\oddsidemargin=0in
\textwidth=6.5in
\textheight=9.0in
\headsep=0.25in
\renewcommand{\thesubsection}{\alph{subsection})}
\renewcommand\lstlistingname{Snippet}
\renewcommand\lstlistlistingname{}

\usepackage{xcolor}
\definecolor{codegreen}{rgb}{0,0.6,0}
\definecolor{codegray}{rgb}{0.5,0.5,0.5}
\definecolor{codeorange}{rgb}{1,0.49,0}
\definecolor{backcolour}{rgb}{0.95,0.95,0.96}

\lstdefinestyle{mystyle}{
    backgroundcolor=\color{backcolour},   
    commentstyle=\color{codegray},
    keywordstyle=\color{codeorange},
    numberstyle=\tiny\color{codegray},
    stringstyle=\color{codegreen},
    basicstyle=\ttfamily\footnotesize,
    breakatwhitespace=false,         
    breaklines=true,                 
    captionpos=b,                    
    keepspaces=true,                 
    numbers=left,                    
    numbersep=5pt,                  
    showspaces=false,                
    showstringspaces=false,
    showtabs=false,                  
    tabsize=2,
    xleftmargin=10pt,
}

\lstset{style=mystyle}
% Title and Author Customization
\title{
    \vspace{3em}
    \textbf{Digital Signal Processing Lab}\\
    Demo 3 - Exercise 4 (Pyaudio, clipping)
    \vspace{1em}
}
\author{
    Saad Zubairi \\ 
    shz2020 \\
    \vspace{1em}
}
\vspace{1em}
\date{September 17th, 2025}

\begin{document}
\maketitle	

\pagebreak

\section*{Solution}

The solution for this was rather simple of course. We can simply calculate the maximum possible value of gain for the specific frame being processed, check it against the user set gain, and if the set gain exceeds the said calculated maximum gain, it should revert to the maxGain value that we calculated.

\begin{lstlisting}[language=python, label={lst:code}, breaklines=true, caption={Snippet of the if function to enable clipping}]
    
    maxGain = ((2**15)/abs(y0))
    if(gain > maxGain):
        print(y0,'/tmax gain:',maxGain)
        gain = maxGain.__floor__() - 1
.
\end{lstlisting}

A small point to note, when calculating the maxGain value, we must use the absolute value of y0 as certain times, low values may lead to integer overflows resulting in unusually high gain values.

Following is the listing for the entire Solution.py file:

\begin{lstlisting}[language=python, label={lst:code}, breaklines=true, caption={Snippet of the if function to enable clipping}]
    # filter_16.py
# 
# Implement the second-order recursive difference equation
# y(n) = x(n) - a1 y(n-1) - a2 y(n-2)
# 
# 16 bit/sample

from math import cos, pi 
import pyaudio
import struct


# Fs : Sampling frequency (samples/second)
Fs = 8000
# Also try other values of 'Fs'. What happens? Why?

T = 1       # T : Duration of audio to play (seconds)
N = T*Fs    # N : Number of samples to play

# Difference equation coefficients
a1 = -1.9
a2 = 0.998

# Initialization
y1 = 0.0
y2 = 0.0
gain = 10178.0
# Also try other values of 'gain'. What is the effect?
# gain = 20000.0

# Create an audio object and open an audio stream for output
p = pyaudio.PyAudio()
stream = p.open(format = pyaudio.paInt16,  
                channels = 1, 
                rate = Fs,
                input = False, 
                output = True)

# paInt16 is 16 bits/sample

# Run difference equation
for n in range(0, N):

    # Use impulse as input signal
    if n == 0:
        x0 = 1.0
    else:
        x0 = 0.0

    # Difference equation
    y0 = x0 - a1 * y1 - a2 * y2

    # Delays
    y2 = y1
    y1 = y0

    maxGain = ((2**15)/abs(y0))
    if(gain > maxGain):
        print(y0,'/tmax gain:',maxGain)
        gain = maxGain.__floor__() - 1
        
    # Output
    output_value = gain * y0
    output_string = struct.pack('h', int(output_value))   # 'h' for 16 bits
    stream.write(output_string)

print("* Finished *")

stream.stop_stream()
stream.close()
p.terminate()

\end{lstlisting}

    
    
\end{document}