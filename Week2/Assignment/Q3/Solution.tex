\documentclass[11pt]{article}
\usepackage{graphicx}
\usepackage{amsmath}
\usepackage{hyperref}
\usepackage{listings}


\topmargin=-0.45in
\evensidemargin=0in
\oddsidemargin=0in
\textwidth=6.5in
\textheight=9.0in
\headsep=0.25in
\renewcommand{\thesubsection}{\alph{subsection})}
\renewcommand\lstlistingname{Snippet}
\renewcommand\lstlistlistingname{}

\usepackage{xcolor}
\definecolor{codegreen}{rgb}{0,0.6,0}
\definecolor{codegray}{rgb}{0.5,0.5,0.5}
\definecolor{codeorange}{rgb}{1,0.49,0}
\definecolor{backcolour}{rgb}{0.95,0.95,0.96}

\lstdefinestyle{mystyle}{
    backgroundcolor=\color{backcolour},   
    commentstyle=\color{codegray},
    keywordstyle=\color{codeorange},
    numberstyle=\tiny\color{codegray},
    stringstyle=\color{codegreen},
    basicstyle=\ttfamily\footnotesize,
    breakatwhitespace=false,         
    breaklines=true,                 
    captionpos=b,                    
    keepspaces=true,                 
    numbers=left,                    
    numbersep=5pt,                  
    showspaces=false,                
    showstringspaces=false,
    showtabs=false,                  
    tabsize=2,
    xleftmargin=10pt,
}

\lstset{style=mystyle}
% Title and Author Customization

% --------------------
% Start from here
% --------------------

\title{
    \vspace{3em}
    \textbf{Digital Signal Processing Lab}\\
    Demo 3 - Exercise 6 (Pyaudio, stereo)
    \vspace{1em}
}
\author{
    Saad Zubairi \\ 
    shz2020 \\
    \vspace{1em}
}
\vspace{1em}
\date{September 17th, 2025}

\begin{document}
\maketitle	

\pagebreak

% --------------------
% Body
% --------------------


\section*{Solution}

To solve this, we just duplicate the values and open two channels.

\begin{itemize}
  \item Duplicate per-sample values when packing (interleave L,R) with two different values for the denominator:
\begin{lstlisting}[language=python]
a1_L = -0.5
a2_L = 0.8

a1_R = -1.9
a2_R = 0.998

y1_L = 0.0
y2_L = 0.0
y1_R = 0.0
y2_R = 0.0
\end{lstlisting}

  \item Since we want two independent channels, compute two y's and pack them using the <hh directive for stereo:
\begin{lstlisting}[language=python]
# left/right states and coeffs exist: (y1_L,y2_L,a1_L,a2_L), (y1_R,y2_R,a1_R,a2_R)
yL = x0 - a1_L*y1_L - a2_L*y2_L
yR = x0 - a1_R*y1_R - a2_R*y2_R
y2_L, y1_L = y1_L, yL
y2_R, y1_R = y1_R, yR
output_string = struct.pack('<hh', int(gain*yL), int(gain*yR))
\end{lstlisting}

  \item Open the stream with two channels:
\begin{lstlisting}[language=python]
stream = p.open(format=pyaudio.paInt16,
                channels=2,      # <-- was 1
                rate=Fs,
                input=False,
                output=True)
\end{lstlisting}
\end{itemize}  
    
Here is the final solution:

\begin{lstlisting}[language=python]
import pyaudio
import struct

Fs = 8000
T = 2
N = T * Fs

a1_L = -0.5
a2_L = 0.8

a1_R = -1.9
a2_R = 0.998

y1_L = 0.0
y2_L = 0.0
y1_R = 0.0
y2_R = 0.0

gain = 5000.0

p = pyaudio.PyAudio()
stream = p.open(format=pyaudio.paInt16,
                channels=2,
                rate=Fs,
                input=False,
                output=True)

for n in range(0, N):
    if n == 0:
        x0 = 1.0
    else:
        x0 = 0.0

    y0_L = x0 - a1_L * y1_L - a2_L * y2_L
    y0_R = x0 - a1_R * y1_R - a2_R * y2_R
    
    y2_L, y1_L = y1_L, y0_L
    y2_R, y1_R = y1_R, y0_R

    output_value_L = gain * y0_L
    output_value_R = gain * y0_R
    output_string = struct.pack('<hh', int(output_value_L), int(output_value_R))
    stream.write(output_string)

print("* Finished *")

stream.stop_stream()
stream.close()
p.terminate()

\end{lstlisting}

\end{document}