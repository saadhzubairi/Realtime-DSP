\documentclass[11pt]{article}
\usepackage{graphicx}
\usepackage{amsmath}
\usepackage{hyperref}
\usepackage{listings}


\topmargin=-0.45in
\evensidemargin=0in
\oddsidemargin=0in
\textwidth=6.5in
\textheight=9.0in
\headsep=0.25in
\renewcommand{\thesubsection}{\alph{subsection})}
\renewcommand\lstlistingname{Snippet}
\renewcommand\lstlistlistingname{}

\usepackage{xcolor}
\definecolor{codegreen}{rgb}{0,0.6,0}
\definecolor{codegray}{rgb}{0.5,0.5,0.5}
\definecolor{codeorange}{rgb}{1,0.49,0}
\definecolor{backcolour}{rgb}{0.95,0.95,0.96}

\lstdefinestyle{mystyle}{
    backgroundcolor=\color{backcolour},   
    commentstyle=\color{codegray},
    keywordstyle=\color{codeorange},
    numberstyle=\tiny\color{codegray},
    stringstyle=\color{codegreen},
    basicstyle=\ttfamily\footnotesize,
    breakatwhitespace=false,         
    breaklines=true,                 
    captionpos=b,                    
    keepspaces=true,                 
    numbers=left,                    
    numbersep=5pt,                  
    showspaces=false,                
    showstringspaces=false,
    showtabs=false,                  
    tabsize=2,
    xleftmargin=10pt,
}

\lstset{style=mystyle}
% Title and Author Customization

% --------------------
% Start from here
% --------------------

\title{
    \vspace{3em}
    \textbf{Digital Signal Processing Lab}\\
    Demo 3 - Exercise 5 (Pyaudio, diff eq)
    \vspace{1em}
}
\author{
    Saad Zubairi \\ 
    shz2020 \\
    \vspace{1em}
}
\vspace{1em}
\date{September 17th, 2025}

\begin{document}
\maketitle	

\pagebreak

% --------------------
% Body
% --------------------


\section*{Solution}

For $h[n] = r^{n}cos(\omega n)z^{-1}u(n)$, using formula:

\begin{center}
    $$
    H(z) = \frac{1-rcos\omega z^{-1}}{1-2rcos\omega z^{-1} + r^{2}z^{-2}}
    $$
\end{center}

\begin{center}
    $$
    H(z) = \frac{b_{0}+b_{1}z^{-1}+b_{2}z^{-2}}{1 + a_{1}z^{-1} + a_{2z^{-2}}}
    $$
\end{center}


Implementing this in code, here is what we can add and change:

\begin{itemize}
    \item Setting the denominator from some arbitrary values of $\omega$ and r.\\
\begin{lstlisting}[language=python, breaklines=true]
r = 0.999
omega = 2*pi*400/Fs      # ~400 Hz tone
a1 = -2 * r * cos(omega)
a2 = r * r
\end{lstlisting}            

    \item We can then add a numerator B(z) with the values for $b_0, b_1, $ and $b_2$\\
\begin{lstlisting}[language=python, breaklines=true]
b0 = 1.0
b1 = a1 / 2.0    # = -r cos(omega)
b2 = 0.0
\end{lstlisting}            

    \item We of course also update the equation:
\begin{lstlisting}[language=python, breaklines=true]
y0 = (b0*x0 + b1*x1 + b2*x2) - a1*y1 - a2*y2
\end{lstlisting}            
    
\item And the variables that need to be updated as well:
\begin{lstlisting}[language=python, breaklines=true]
x2, x1 = x1, x0
    y2, y1 = y1, y0

\end{lstlisting}            

\item Lastly, to keep the gain in check, we can reuse the same if statement used in the previous question
\begin{lstlisting}[language=python, breaklines=true]
maxGain = ((2**15)/abs(y0)).__floor__() - 1
    if(gain > maxGain):
        print(y0,'/tmax gain:',maxGain)
        gain = maxGain
\end{lstlisting}           

\end{itemize}
% --------------------
% Code Blocks example
% --------------------
\pagebreak
The final code is given as follows:
\begin{lstlisting}[language=python, breaklines=true]
from math import cos, pi
import pyaudio, struct

Fs = 8000
T = 1
N = T * Fs

r = 0.999
omega = 2*pi*400/Fs      # ~400 Hz tone
a1 = -2 * r * cos(omega)
a2 = r * r

# h[n] = r^n cos(omega n) u[n]
b0 = 1.0
b1 = a1 / 2.0    # = -r cos(omega)
b2 = 0.0

# maintaing states
x1 = x2 = 0.0
y1 = y2 = 0.0

gain = 10000.0

p = pyaudio.PyAudio()
stream = p.open(format=pyaudio.paInt16, channels=1, rate=Fs, input=False, output=True)

for n in range(N):
    # Use impulse as input signal
    if n == 0:
        x0 = 1.0
    else:
        x0 = 0.0


    y0 = (b0*x0 + b1*x1 + b2*x2) - a1*y1 - a2*y2
    
    # Delays
    x2, x1 = x1, x0
    y2, y1 = y1, y0

    maxGain = ((2**15)/abs(y0)).__floor__() - 1
    if(gain > maxGain):
        print(y0,'/tmax gain:',maxGain)
        gain = maxGain
        
    output_value = gain * y0
    output_string = struct.pack('h', int(output_value)) 
    stream.write(output_string)

print("* Finished *")
stream.stop_stream(); stream.close(); p.terminate()

\end{lstlisting}           

    
\end{document}