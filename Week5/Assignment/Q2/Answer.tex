\documentclass[11pt]{article}
\usepackage{graphicx}
\usepackage{amsmath}
\usepackage{hyperref}
\usepackage{listings}


\topmargin=-0.45in
\evensidemargin=0in
\oddsidemargin=0in
\textwidth=6.5in
\textheight=9.0in
\headsep=0.25in
\renewcommand{\thesubsection}{\alph{subsection})}
\renewcommand\lstlistingname{Snippet}
\renewcommand\lstlistlistingname{}

\usepackage{xcolor}
\definecolor{codegreen}{rgb}{0,0.6,0}
\definecolor{codegray}{rgb}{0.5,0.5,0.5}
\definecolor{codeorange}{rgb}{1,0.49,0}
\definecolor{backcolour}{rgb}{0.95,0.95,0.96}

\lstdefinestyle{mystyle}{
    backgroundcolor=\color{backcolour},   
    commentstyle=\color{codegray},
    keywordstyle=\color{codeorange},
    numberstyle=\tiny\color{codegray},
    stringstyle=\color{codegreen},
    basicstyle=\ttfamily\footnotesize,
    breakatwhitespace=false,         
    breaklines=true,                 
    captionpos=b,                    
    keepspaces=true,                 
    numbers=left,                    
    numbersep=5pt,                  
    showspaces=false,                
    showstringspaces=false,
    showtabs=false,                  
    tabsize=2,
    xleftmargin=10pt,
}

\lstset{style=mystyle}
% Title and Author Customization

% --------------------
% Start from here
% --------------------

\title{
    \vspace{3em}
    \textbf{Digital Signal Processing Lab}\\
    Demo 52 - Exercise 1 (Bandpass filter with real-time plotting)
    \vspace{1em}
}
\author{
    Saad Zubairi \\ 
    shz2020 \\
    \vspace{1em}
}
\vspace{1em}
\date{\today}
\begin{document}
\maketitle	

\pagebreak

% --------------------
% Body
% --------------------


\section*{Solution}

To solve this, we can start with the \texttt{demo 52 - plotting audio\\prog\_05.py} file included with the demo as it contains the animation functions implementation and take the recursive filter from the \texttt{demo 06 - filter wave file\\wave\_filter\_python.py} file.
\\
\\
The following are the additions and changes made the file for this implementation:

\begin{itemize}
    \item The filter specifications are implemented wthe constants as follows:
    \begin{lstlisting}[language=python, label={lst:code}, breaklines=true, caption={Filter initialization}]
    # Bandpass Filter Coefficients
b0 =  0.008442692929081
b2 = -0.016885385858161
b4 =  0.008442692929081

a1 = -3.580673542760982
a2 =  4.942669993770672
a3 = -3.114402101627517
a4 =  0.757546944478829

# Initialization of Delay Elements
x1 = 0.0
x2 = 0.0
x3 = 0.0
x4 = 0.0
y1 = 0.0
y2 = 0.0
y3 = 0.0
y4 = 0.0
    \end{lstlisting}        

    \item The block level modulation for the signal is changed to incorporate the filter application:
    \begin{lstlisting}[language=python, label={lst:code}, breaklines=true, caption={Block level output processing}]
    for n in range(BLOCKLEN):
        x0 = input_block[n]

        y0 = b0*x0 + b2*x2 + b4*x4 - a1*y1 - a2*y2 - a3*y3 - a4*y4

        # Update delays
        x4, x3, x2, x1 = x3, x2, x1, x0
        y4, y3, y2, y1 = y3, y2, y1, y0

        # Clip to 16-bit
        output_block[n] = int(clip16(y0))
    \end{lstlisting}   
    
\end{itemize}


Most of the other implementation largely remains the same. Since this is filter's response is LTI, and the coefficients are constants, we do not need to incorporate \texttt{theta} in this.
\\
\\
The demo video is attached with this assignment as 

% \href{run:./Answer.mp4}{Open Demo Video}



\pagebreak
\section{Addendum}
Here's the full code for the solution:

\begin{lstlisting}[language=python, label={lst:code}, breaklines=true, caption={Full implementation}]
import pyaudio
import struct
import wave
import matplotlib
from matplotlib import pyplot
from matplotlib import animation
import math

def clip16( x ):    
    # Clipping for 16 bits
    if x > 32767:
        x = 32767
    elif x < -32768:
        x = -32768
    else:
        x = x        
    return (x)

matplotlib.use('TkAgg')
# matplotlib.use('MacOSX')

print('The matplotlib backend is %s' % pyplot.get_backend())      # Plotting backend

# Specify wave file
import os
wavefile = os.path.join(os.path.dirname(__file__), 'author.wav')
wf = wave.open(wavefile, 'rb')

# Read wave file properties
RATE        = wf.getframerate()     # Frame rate (frames/second)
WIDTH       = wf.getsampwidth()     # Number of bytes per sample
LEN         = wf.getnframes()       # Signal length
CHANNELS    = wf.getnchannels()     # Number of channels

print('The file has %d channel(s).'         % CHANNELS)
print('The file has %d frames/second.'      % RATE)
print('The file has %d frames.'             % LEN)
print('The file has %d bytes per sample.'   % WIDTH)

# Bandpass Filter Coefficients
b0 =  0.008442692929081
b2 = -0.016885385858161
b4 =  0.008442692929081

a1 = -3.580673542760982
a2 =  4.942669993770672
a3 = -3.114402101627517
a4 =  0.757546944478829

# Initialization of Delay Elements
x1 = 0.0
x2 = 0.0
x3 = 0.0
x4 = 0.0
y1 = 0.0
y2 = 0.0
y3 = 0.0
y4 = 0.0

# Audio Parameters
BLOCKLEN = 256
BLOCK_DURATION = 1000.0 * BLOCKLEN / RATE  # duration in milliseconds
print('Block length: %d' % BLOCKLEN)
print('Duration of block in milliseconds: %.2f' % BLOCK_DURATION)

# Audio Stream Setup
p = pyaudio.PyAudio()
PA_FORMAT = p.get_format_from_width(WIDTH)

stream = p.open(
    format = PA_FORMAT,
    channels = CHANNELS,
    rate = RATE,
    input = False,
    output = True,
    frames_per_buffer = BLOCKLEN)

# Plot Setup
fig1 = pyplot.figure(1)
fig1.set_figwidth(8.0)
fig1.set_figheight(6.0)

ax1 = fig1.add_subplot(2, 1, 1)
ax2 = fig1.add_subplot(2, 1, 2)

[g1] = ax1.plot([], [])
[g2] = ax2.plot([], [])

def my_init():
    g1.set_xdata([1000 * i / RATE for i in range(BLOCKLEN)])
    g1.set_ydata(BLOCKLEN * [0])
    ax1.set_ylim(-32000, 32000)
    ax1.set_xlim(0, 1000 * BLOCKLEN / RATE)
    ax1.set_xlabel('Time (milliseconds)')
    ax1.set_title('Input Signal')

    g2.set_xdata([1000 * i / RATE for i in range(BLOCKLEN)])
    g2.set_ydata(BLOCKLEN * [0])
    ax2.set_ylim(-32000, 32000)
    ax2.set_xlim(0, 1000 * BLOCKLEN / RATE)
    ax2.set_xlabel('Time (milliseconds)')
    ax2.set_title('Output Signal (Bandpass Filtered)')

    return (g1, g2)

# Animation Update Function
def my_update(i):
    global x1, x2, x3, x4, y1, y2, y3, y4

    input_bytes = wf.readframes(BLOCKLEN)

    # Rewind if end of file
    if len(input_bytes) < WIDTH * BLOCKLEN:
        wf.rewind()
        input_bytes = wf.readframes(BLOCKLEN)

    input_block = struct.unpack('h' * BLOCKLEN, input_bytes)
    output_block = [0] * BLOCKLEN

    # Filter Processing (Recursive)
    for n in range(BLOCKLEN):
        x0 = input_block[n]

        y0 = b0*x0 + b2*x2 + b4*x4 - a1*y1 - a2*y2 - a3*y3 - a4*y4

        # Update delays
        x4, x3, x2, x1 = x3, x2, x1, x0
        y4, y3, y2, y1 = y3, y2, y1, y0

        # Clip to 16-bit
        output_block[n] = int(clip16(y0))

    g1.set_ydata(input_block)
    g2.set_ydata(output_block)
    output_bytes = struct.pack('h' * BLOCKLEN, *output_block)
    stream.write(output_bytes, BLOCKLEN)

    return (g1, g2)

my_anima = animation.FuncAnimation(
    fig1,
    my_update,
    init_func = my_init,
    interval = 10,
    blit = True,
    cache_frame_data = False,
    repeat = False)

fig1.tight_layout()
pyplot.show()

stream.stop_stream()
stream.close()
p.terminate()
wf.close()

print('* Finished')
    \end{lstlisting}   


\end{document}